\documentclass[a4paper,11pt,DIV=13]{scrartcl}
\usepackage[scaled]{helvet}
\usepackage[utf8]{inputenc} 
\usepackage[T1]{fontenc}
\usepackage{graphicx,textcomp,booktabs}
\usepackage{amsmath,mathptmx,courier,mhchem,siunitx}
\usepackage[usenames,dvipsnames]{xcolor}
\usepackage[%  
  bookmarks,                   %% PDF-Lesezeichen
  bookmarksopen=true,          %% Lesezeichenbaum aufgeklappt...
  bookmarksopenlevel=2,        %% ...um eine Ebene
  bookmarksnumbered=true,      %% Lesezeichen numerieren
  pdfusetitle,                 %% LaTeX-Titelei als Metainfo nehmen
  pdfstartpage={1},            %% mit welcher Seite das PDF öffnen
  pdfstartview={FitH},         %% Zoom auf Seitenbreite
  pdfkeywords={},              %% Stichwörter fürs PDF, kommagetrennt
  pdfsubject={},               %% Themenbeschreibung kurz
  pdfcreator={LaTeX with KOMA-Script and hyperref package},
  hyperfootnotes=true,         %% Links auf Fußnoten
  hyperindex=true,             %% Indexeinträge verweisen auf Text
   colorlinks = true,
  linkcolor =MidnightBlue, %black
  anchorcolor = black,
	citecolor=  ForestGreen,
	filecolor= magenta,
	menucolor = blue ,	
	urlcolor= cyan, 
  linkbordercolor={0 1 1},     %% Rahmenfarbe interne Links
  menubordercolor={0 1 1},     %% Rahmenfarbe Literaturlinks
  urlbordercolor={1 0 0}       %% Rahmenfarbe externe Links
]{hyperref}                    %% Hyperlinks und Lesezeichen in PDF 
\usepackage[all]{hypcap}

\begin{document}
\begin{centering}
{\LARGE ES-IBD Explorer}\\
\vspace{.5cm}
{\Large A comprehensive data acquisition and analysis tool for \\ Electrospray Ion-Beam Deposition experiments.} \\
\vspace{.5cm}
{\today}\\
\end{centering}

\section{Purpose and Concept}

The ES-IBD Explorer controls all aspects of a Electrospray Ion-Beam deposition (ESIBD, also known as soft landing or preparative mass spectrometry) experiment, including optimization and characterization of the ion-beam energy, intensity, and size, as well as monitoring of the deposition. At each step, results and metadata are saved to document and the experiment and allow for reproduction of all experimental conditions.

Just like modern apps on your phone the ES-IBD Explorer was designed to be used intuitively and without the need to memorize a lengthy manual. The main functionality should be self-evident from the user interface (see tooltips!). However, as this software is made for a very specific scientific application, there are some aspects worth discussing in more detail to make sure you can make the most use of it. Please get in touch if you have any feature requests or bug reports. The ES-IBD Explorer should allow researchers to focus on the science, instead of working around bugs and incompatibilities.

ES-IBD Explorer is implemented based on Python and PyQt and optimized for Microsoft Windows.

In the following the main features are explained as organized in the user interface.

\section{User Interface}
The user interface is structured into four main sections. The tabs in the lower left contain all controls for \textbf{Configuration and Management} of measurement modes and outputs. During and after the measurement, results are displayed using various \textbf{Displays} in the lower right. The most important signal of any deposition experiment, the real-time ion-current, is always visible in a dedicated \textbf{Current Display} on the top. Finally, there is a \textbf{Console} for status messages and advanced features that can be opened by dragging from the bottom. The content and state of the user interface is restored when restarting the software allowing users to carry on directly where they left. All devices will be disconnected and turned of to leave them in a save state when the software is closed. All numeric inputs can be changed using the arrow keys, like in LabVIEW.

\subsection{Configuration and Management}

\subsubsection{Settings}
Allows to edit, save, and load all general program and hardware settings. 
Settings can be edited either directly or from the context menu that opens on right click.
Settings are stored in a .ini file which can be edited with any text editor if needed.
The settings file that is used on startup is automatically generated if it does not exist.
The most important parameters define where the software configuration and data is stored. 
These parameters are saved in the registry and will be restored even if the .ini file has to be replaced for any reason.

All results are automatically saved to the session path, based on the date, time, substrate, ion, session type, and automatically incrementing measurement number.
This ensures a consistent file structure that makes it easy to find related sessions. 
The session path is updated automatically when any of its elements is changed or the program is restated. 

\subsubsection{Voltage}
Contains an array of all voltages channels. While ES-IBD Explorer was initially developed to support power supply modules based on ISEG ECH244, other instruments can be mapped onto this list if needed. The configuration is saved as a .ini file. Importing the file recreates the entire configuration, while loading only updated the voltages. The voltages are monitored and a warning is given if the set potentials are not reached. Each channel can be selected to obtain a slider which enables intuitive optimization on the current signals. The potentials can also be visualized. Channels can be selected for automatic optimization with a genetic algorithm (GA). Make sure to choose save voltage limits and obtain a stable current, before starting the optimization. The initial parameters can always be restored in case the optimization fails. The advanced view reveals additional options that are typically not changed after initial configuration. Notably, these options allow to define virtual channels that allow to control a group of potentials with a single parameter using equations.

\subsubsection{Current}
Contains an array of all current channels. While ES-IBD Explorer was initially developed to support RBD 9103 picoammeters, other instruments can be mapped onto this list if needed. The configuration is saved as a .ini file. Background signals, defined manually or based on the current value, can be subtracted. The accumulated charge is proportional to the number of deposited ions. It can also reveal on which elements ions are lost. The current history can be exported to the session folder for documentation. The advanced view reveals additional options that are typically not changed after initial configuration. Notably, these options allow to define virtual channels, e.g., to monitor the transmission, defined as the difference between an emitter and capillary current.

\subsubsection{2D Scan}
Records the current on one element as a function of two voltage channels, typically deflectors. All active current channels are recorded but only the selected channel is shown. The recorded data is interpolated after the scan to enhance the visualization. The resulting image is used to identify elements like apertures, samples, and detectors. The beam can be moved between those elements by clicking and dragging on the image while holding down the Ctrl key. Note that the currently selected channels might deviate from those used to record an image. Scan limits can be adopted from the figure, e.g., to record a more detailed scan after zooming in on a feature. Scans are saved in the widely used HDF5 file format that allows to keep data and metadata together in a structured binary file. External viewers, such as HDFView, or minimal python scripts based on the h5py package can be used if files need to be accessed outside of the ES-IBD Explorer.

\subsubsection{Energy}
Records the current on one element, typically a detector plate, as a function of one potential, typically a retarding grid. All active current channels are recorded but only the selected channel is shown. The resulting image contains the measured transmission data, a discrete derivative, and a Gaussian fit that reveals beam-energy center and width. The potential on the selected channel can be changed by clicking and dragging on the image while holding down the Ctrl key. Again, scans are saved in the HDF5 file format.

\subsubsection{Depo}
Records the sample current and accumulated charge during deposition. A target carge can be defined. If the current drops below the warning threshold a sound will be emitted. Again, scans are saved in the HDF5 file format. As data analysis is decoupled from data acquisition, you can continue to use all other scan modes and optimization while the deposition recording is running.

\subsubsection{Explorer}
The integrated explorer is used to navigate and visualize all results. 
All files can be accessed using the Windows Explorer, e.g., when working on a computer where ES-IBD Explorer is not installed. 
However, the integrated explorer provides dedicated displays for all files that were created with ES-IBD Explorer, its LabVIEW based predecessor ES-IBD Control, and a number of other common file types including .tex, .tex, .png, .jpg, .svg, .pdf, .h5, .hdf5, .star, .pdb1, .css, .py, .js, .html, .ini, .bat, etc. These displays were made to simplify data analysis and documentation. They allow saving images as files or sending them to the clipboard (Edit/Copy display to clipboard). Right clicking on many files opens a context menu that allows to load settings directly into ES-IBD Explorer. A double click will open the file an external program.

\subsubsection{Custom Extensions}
The minimal code in custom.py demonstrates how to integrate your own custom elements to the ES-IBD Explorer. You will find a line in the combobox of the console that runs this code to add a custom tab with a button. This should be sufficient as long as your code does not requires interaction with any other elements of the ES-IBD Explorer. Please get in touch if this is not sufficient for your application.

\subsubsection{Custom Device Extensions}
TODO
summarize all new features
The modules for communication with ISEG power supply and RBD picoammeters are now based on the same classes. This means they can be used as an additional example of how to integrate devices. If your instrument does not include these devices, simply remove them by commenting out the line that is initializing them.
Custom devices can now provide associated settings by overwriting the getDefaultSettings method. The settings will be grouped and shown in the general settings tab.

\subsection{Displays}
A variety of displays are implemented and typically selected automatically based on the file type. Images come with standard matplotlib toolbars for zooming, saving, etc. If an image is displayed, the Text tab may contain a useful representation of the corresponding file contents. For some formats data is also shown using pyqtgraph controls which contain complementary functionality. The Notes tab can be used to add quick comments to a session which are saved in simple text files, but are not intended to replace a lab book. When looking at mass spectra, clicking on peaks in a charge state series while holding down the Ctrl key provides a quick estimate of charge state and mass. The detailed results are shown in the console, and help to evaluate the quality of the estimate.

\subsection{Current Display}
Displays the real-time current measurements. The length of the displayed history is determined by the Display Time setting. If a voltage channel is selected, the x axis will switch from displaying the time to displaying the potential of the selected channel and a corresponding slider will become available. A double click on the purple bar on the left allows to pop out the Current Display, e.g., to show it maximized on a second monitor. 

\subsection{Console}
The console should typically not be needed unless something goes wrong. Status messages will be logged here.
All features implemented in the user interface and more can be accessed directly from this console.
It is mainly intended for debugging. Use at your own Risk! You can select some commonly used commands directly from the combobox.


\section{Change Log}


\subsection*{Version 0.4}
Update notes: you need to delete settings.ini and config.h5 from your settings folder. Updated versions will be generated on the first start after the update. You can then manualy restore your settings. In the future, file format changes during updates will be automated. 
In this release, the definition of lens and name in the current tab were changed to name and deviceName to increase consistency. Please adjust your settings accordingly after the first start.
\paragraph{New features}
\begin{itemize}
\item Added "Depo" tab to display and document deposition.
\item Result of mass analysis is shown in plot.
\item Added "Move to Recycle Bin" feature to explorer items.
\item The initial signal is used instead of 0 to initialize 2D scans, making to displayed color range more usfull.
\item The device configuration can be exported directly to the current session.
\item The most common current plot functions are now directly attached to the current plot and always accessible. In addition, the voltage button moved here to indicate the voltage status at all times and allow quick access in case of an emergency.
\item The position on the 2D scan is indicated by a cursor.
\item The manual can be accessed from the Help menu.
\item Combo boxes can no longer change accidentally by mouse wheel scrolling.
\item The optimization progress is shown in a dedicated tab while running.
\item Background subtraction is active by default, though there is no effect until backgrounds are defined.
\item Color bar for 2D scans is labeled.
\item More content is dockable.
\item Added custom device tab example. See new section in documentation for details.
\item Devices will now generate default .ini files if no file is found. This can be useful to make sure the files are valid before populating them with additional channels.
\item Settings are saved directly, not just when exiting program.
\item The Testmode can now be activated directly from the settings menu and stays active if the application is restarted
\item All device settings is saved together with the measurement data in a single .h5 file. Values can be loaded by right clicking the file or importing from the corresponding device tab.
\item The performance and error messages of the equation evaluation routine have been improved. Operators in equations need to be separated from the channel names by spaces.
\item 
\end{itemize}

\paragraph{Bug fixes}
\begin{itemize}
\item Fixed live update of min max limits for voltage channels.
\item Scans cannot be started before the previous scan has been saved.
\item Made sure that potentials are reset to initial values before scans are saved.
\item Displays and channels are no longer overwritten when loading scan settings. Instead, only non existing items from the file are added to the present list.
\item Current channels are reset and removed from plot when disabled by the user.
\item The file labels on top of figures are scaled to the figure width to be always visible.
\item Changing the display channel during and after scanning is now more reliable.
\item Mass spec data display interpolates data before down sampling for more accurate display.
\item 
\end{itemize}

\paragraph{Performance improvements}
\begin{itemize}
\item Explorer will only index files as the tree is expanded.
\item
\end{itemize}


\end{document}

